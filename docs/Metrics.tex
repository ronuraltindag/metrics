\documentclass[]{book}
\usepackage{lmodern}
\usepackage{amssymb,amsmath}
\usepackage{ifxetex,ifluatex}
\usepackage{fixltx2e} % provides \textsubscript
\ifnum 0\ifxetex 1\fi\ifluatex 1\fi=0 % if pdftex
  \usepackage[T1]{fontenc}
  \usepackage[utf8]{inputenc}
\else % if luatex or xelatex
  \ifxetex
    \usepackage{mathspec}
  \else
    \usepackage{fontspec}
  \fi
  \defaultfontfeatures{Ligatures=TeX,Scale=MatchLowercase}
\fi
% use upquote if available, for straight quotes in verbatim environments
\IfFileExists{upquote.sty}{\usepackage{upquote}}{}
% use microtype if available
\IfFileExists{microtype.sty}{%
\usepackage[]{microtype}
\UseMicrotypeSet[protrusion]{basicmath} % disable protrusion for tt fonts
}{}
\PassOptionsToPackage{hyphens}{url} % url is loaded by hyperref
\usepackage[unicode=true]{hyperref}
\hypersetup{
            pdftitle={EC282: Homework Assignments},
            pdfauthor={Onur Altındağ},
            pdfborder={0 0 0},
            breaklinks=true}
\urlstyle{same}  % don't use monospace font for urls
\usepackage[margin=2cm]{geometry}
\usepackage{natbib}
\bibliographystyle{apalike}
\usepackage{color}
\usepackage{fancyvrb}
\newcommand{\VerbBar}{|}
\newcommand{\VERB}{\Verb[commandchars=\\\{\}]}
\DefineVerbatimEnvironment{Highlighting}{Verbatim}{commandchars=\\\{\}}
% Add ',fontsize=\small' for more characters per line
\usepackage{framed}
\definecolor{shadecolor}{RGB}{248,248,248}
\newenvironment{Shaded}{\begin{snugshade}}{\end{snugshade}}
\newcommand{\KeywordTok}[1]{\textcolor[rgb]{0.13,0.29,0.53}{\textbf{#1}}}
\newcommand{\DataTypeTok}[1]{\textcolor[rgb]{0.13,0.29,0.53}{#1}}
\newcommand{\DecValTok}[1]{\textcolor[rgb]{0.00,0.00,0.81}{#1}}
\newcommand{\BaseNTok}[1]{\textcolor[rgb]{0.00,0.00,0.81}{#1}}
\newcommand{\FloatTok}[1]{\textcolor[rgb]{0.00,0.00,0.81}{#1}}
\newcommand{\ConstantTok}[1]{\textcolor[rgb]{0.00,0.00,0.00}{#1}}
\newcommand{\CharTok}[1]{\textcolor[rgb]{0.31,0.60,0.02}{#1}}
\newcommand{\SpecialCharTok}[1]{\textcolor[rgb]{0.00,0.00,0.00}{#1}}
\newcommand{\StringTok}[1]{\textcolor[rgb]{0.31,0.60,0.02}{#1}}
\newcommand{\VerbatimStringTok}[1]{\textcolor[rgb]{0.31,0.60,0.02}{#1}}
\newcommand{\SpecialStringTok}[1]{\textcolor[rgb]{0.31,0.60,0.02}{#1}}
\newcommand{\ImportTok}[1]{#1}
\newcommand{\CommentTok}[1]{\textcolor[rgb]{0.56,0.35,0.01}{\textit{#1}}}
\newcommand{\DocumentationTok}[1]{\textcolor[rgb]{0.56,0.35,0.01}{\textbf{\textit{#1}}}}
\newcommand{\AnnotationTok}[1]{\textcolor[rgb]{0.56,0.35,0.01}{\textbf{\textit{#1}}}}
\newcommand{\CommentVarTok}[1]{\textcolor[rgb]{0.56,0.35,0.01}{\textbf{\textit{#1}}}}
\newcommand{\OtherTok}[1]{\textcolor[rgb]{0.56,0.35,0.01}{#1}}
\newcommand{\FunctionTok}[1]{\textcolor[rgb]{0.00,0.00,0.00}{#1}}
\newcommand{\VariableTok}[1]{\textcolor[rgb]{0.00,0.00,0.00}{#1}}
\newcommand{\ControlFlowTok}[1]{\textcolor[rgb]{0.13,0.29,0.53}{\textbf{#1}}}
\newcommand{\OperatorTok}[1]{\textcolor[rgb]{0.81,0.36,0.00}{\textbf{#1}}}
\newcommand{\BuiltInTok}[1]{#1}
\newcommand{\ExtensionTok}[1]{#1}
\newcommand{\PreprocessorTok}[1]{\textcolor[rgb]{0.56,0.35,0.01}{\textit{#1}}}
\newcommand{\AttributeTok}[1]{\textcolor[rgb]{0.77,0.63,0.00}{#1}}
\newcommand{\RegionMarkerTok}[1]{#1}
\newcommand{\InformationTok}[1]{\textcolor[rgb]{0.56,0.35,0.01}{\textbf{\textit{#1}}}}
\newcommand{\WarningTok}[1]{\textcolor[rgb]{0.56,0.35,0.01}{\textbf{\textit{#1}}}}
\newcommand{\AlertTok}[1]{\textcolor[rgb]{0.94,0.16,0.16}{#1}}
\newcommand{\ErrorTok}[1]{\textcolor[rgb]{0.64,0.00,0.00}{\textbf{#1}}}
\newcommand{\NormalTok}[1]{#1}
\usepackage{longtable,booktabs}
% Fix footnotes in tables (requires footnote package)
\IfFileExists{footnote.sty}{\usepackage{footnote}\makesavenoteenv{long table}}{}
\usepackage{graphicx,grffile}
\makeatletter
\def\maxwidth{\ifdim\Gin@nat@width>\linewidth\linewidth\else\Gin@nat@width\fi}
\def\maxheight{\ifdim\Gin@nat@height>\textheight\textheight\else\Gin@nat@height\fi}
\makeatother
% Scale images if necessary, so that they will not overflow the page
% margins by default, and it is still possible to overwrite the defaults
% using explicit options in \includegraphics[width, height, ...]{}
\setkeys{Gin}{width=\maxwidth,height=\maxheight,keepaspectratio}
\IfFileExists{parskip.sty}{%
\usepackage{parskip}
}{% else
\setlength{\parindent}{0pt}
\setlength{\parskip}{6pt plus 2pt minus 1pt}
}
\setlength{\emergencystretch}{3em}  % prevent overfull lines
\providecommand{\tightlist}{%
  \setlength{\itemsep}{0pt}\setlength{\parskip}{0pt}}
\setcounter{secnumdepth}{5}
% Redefines (sub)paragraphs to behave more like sections
\ifx\paragraph\undefined\else
\let\oldparagraph\paragraph
\renewcommand{\paragraph}[1]{\oldparagraph{#1}\mbox{}}
\fi
\ifx\subparagraph\undefined\else
\let\oldsubparagraph\subparagraph
\renewcommand{\subparagraph}[1]{\oldsubparagraph{#1}\mbox{}}
\fi

% set default figure placement to htbp
\makeatletter
\def\fps@figure{htbp}
\makeatother

\usepackage{booktabs}

\title{EC282: Homework Assignments}
\author{Onur Altındağ}
\date{Last update: 2020-01-08}

\begin{document}
\maketitle

{
\setcounter{tocdepth}{1}
\tableofcontents
}
\chapter{General Rules and
Principles}\label{general-rules-and-principles}

\section{Installing R and RStudio}\label{installing-r-and-rstudio}

Here are the
\href{https://courses.edx.org/courses/UTAustinX/UT.7.01x/3T2014/56c5437b88fa43cf828bff5371c6a924/}{instructions}
for installing R and RStudio on your Windows or Mac desktop. Skip the
third part and do not install ``SDSFoundations Package''.

\section{Basic rules and best
practices}\label{basic-rules-and-best-practices}

All files should exist in a local folder that syncs to a cloud-storage
service. No file you ever work on should be at risk of being lost if
your computer ceases to function or be in your possesion. NEVER place
any file on ``downloads'' or ``desktop'' folders.

Get a free cloud-storage service with a desktop application that syncs
to a cloud-storage service. I like the
\href{https://help.dropbox.com/installs-integrations/desktop/desktop-application-overview}{Dropbox
desktop app} but feel free to choose any other service. You don't need a
lot of space so free version of any desktop cloud app would work. Under
the Dropbox folder, create a designated folder for this course such as
EC282.

All subfolders under EC282 and files in them should have unique and
descriptive construction: DON'T use spaces in file or folder names. Here
is an example of a folder structure that might work for a student in
this class:

\begin{verbatim}
EC282
│  
│
└───Course_docs
│   │    SyllabusEC282.pdf
│   │    LectureNotes.pdf 
└───Assignments
│   └───Assignment1
│       │   dataset1name.Rda
│       │   Lastname_Firstname_Assignment1_EC282.R
│   └───Assignment2
│       │   dataset2name.Rda
│       │   Lastname_Firstname_Assignment1_EC282.R
│   └───Assignment3
│       │   dataset3name.Rda
│       │   Lastname_Firstname_Assignment3_EC282.R
│   └───Assignment4
│       │   dataset4name.Rda
│       │   Lastname_Firstname_Assignment4_EC282.R
│       │   ...
└───Exams
│   └───Midterm1
│       │   Midterm1Review.pdf
│       │   Midterm1Review_myanswers.docx
|       |   ...
│   
\end{verbatim}

\section{Header}\label{header}

At the beginning of any R script, you should have a standard header that
you use across all scripts that clears the workspace, loads/installs
packages as necessary, sets the working directory, etc. Here is an
example that you can copy paste to the header of any script that you
use:

\begin{Shaded}
\begin{Highlighting}[]
\NormalTok{###############################################################################}
\CommentTok{# list the packages we need and loads them, installs them automatically if we don't have them}
\CommentTok{# add any package that you need to the list  }
\NormalTok{need <-}\StringTok{ }\KeywordTok{c}\NormalTok{(}\StringTok{'glue'}\NormalTok{, }\StringTok{'dplyr'}\NormalTok{,}\StringTok{'readxl'}\NormalTok{, }\StringTok{'MASS'}\NormalTok{, }\StringTok{'ggplot2'}\NormalTok{,}\StringTok{'tidyr'}\NormalTok{,}\StringTok{'AER'}\NormalTok{,}\StringTok{'scales'}\NormalTok{,}\StringTok{'mvtnorm'}\NormalTok{, }
          \StringTok{'stargazer'}\NormalTok{,}\StringTok{'httr'}\NormalTok{)}

\NormalTok{have <-}\StringTok{ }\NormalTok{need }\OperatorTok\StringTok{ }\KeywordTok{rownames}\NormalTok{(}\KeywordTok{installed.packages}\NormalTok{()) }
\ControlFlowTok{if}\NormalTok{(}\KeywordTok{any}\NormalTok{(}\OperatorTok{!}\NormalTok{have)) }\KeywordTok{install.packages}\NormalTok{(need[}\OperatorTok{!}\NormalTok{have]) }
\KeywordTok{invisible}\NormalTok{(}\KeywordTok{lapply}\NormalTok{(need, library, }\DataTypeTok{character.only=}\NormalTok{T)) }

\CommentTok{# To set up the working directory}
\KeywordTok{getwd}\NormalTok{()}
\KeywordTok{setwd}\NormalTok{(}\KeywordTok{getwd}\NormalTok{()) }\CommentTok{#change getwd() here is you need to set a different working directory}


\CommentTok{#this clears the workspace}
\KeywordTok{rm}\NormalTok{(}\DataTypeTok{list =} \KeywordTok{ls}\NormalTok{()) }
\CommentTok{#this sets the random number generator seed to your birthday for replication}
\KeywordTok{set.seed}\NormalTok{(}\DecValTok{01011999}\NormalTok{)}
\NormalTok{###############################################################################}
\end{Highlighting}
\end{Shaded}

When coding, use relative references to files. Typically, any script
will begin looking for files in the working directory. At any time you
can type \texttt{getwd()} on your Rstudio console to see the current
working directory. The header above automatically sets the working
directory to the folder that the R script is included. For example, if
you are working on \texttt{Lastname\_Firstname\_Assignment1\_EC282.R}
script and need to load file \texttt{dataset1name.Rda} into an object,
then you would simply run:

\begin{Shaded}
\begin{Highlighting}[]
\KeywordTok{load}\NormalTok{(dataset1name.Rda)}
\end{Highlighting}
\end{Shaded}

However, if you were working in the same .R file, and needed to access
\texttt{dataset2name.Rda}, you would need to point the program to a
directory outside the current working directory -- so, you go up one
level, over one folder, and look there:

\begin{Shaded}
\begin{Highlighting}[]
\KeywordTok{load}\NormalTok{(..}\OperatorTok{/}\NormalTok{Assignment2}\OperatorTok{/}\NormalTok{dataset2name.Rda)}
\end{Highlighting}
\end{Shaded}

When learning R, the most important skill that you need to acquire is to
be able to \textbf{google} your problem. There is probably not a single
R question that you have yet has not been answered on
\href{https://stackoverflow.com/}{Stack Overflow}.

\section{How to submit the homework
assignment}\label{how-to-submit-the-homework-assignment}

You can use a snipping tool to copy and paste the relevant output and
figures from RStudio console to a word file, type the answers, save
everything and upload it on BlackBoard/Assignments. If you want to have
a more elegant looking homework output the \texttt{stargazer} package is
very powerful in transforming your analysis into publishable formats.

\chapter{Homework Assigment I}\label{homework-assigment-i}

\textbf{Deadline}: Feb 16, 2020

\textbf{Source:} Stock and Watson, 3rd Updated Edition. Exercise 3.1

\textbf{Data description:} You can find the data description
\href{https://wps.pearsoned.com/wps/media/objects/11422/11696965/empirical/empex_tb/CPS92_08_Description.pdf}{here}.

\textbf{Question I}

\textbf{a.} Compute the sample mean for average hourly earnings
(\texttt{ahe}) in 1992 and 2012. Construct a 95\% confidence internal
for the population means of \texttt{ahe} in 1992 and 2008 and the change
between 1992 and 2008.

\textbf{b.} In 2008, the values of the Consumer Price Index (CPI) was
215.2. In 1992, the value of the CPI was 140.3. Repeat (a) but use AHE
measured in real 2008 dollars (\$2008); that is, adjust the 1992 data
for the price inflation that occured between 1992 and 2008.

\textbf{c.} If you were interested in the change in workers' purchasing
power from 1992 to 2008, would you use the results from (a) or (b)?
Explain.

\textbf{d.} Use the 2008 data to construct a 95\% confidence internal
for the mean of \texttt{ahe} for workers with a college degree.
Construct a 95\% confidence interval for the difference between the two
means.

\textbf{e.} Repeat (d) using the 1992 data expressed in \$2008.

\textbf{f.} Did real (inflation-adjusted) wages of high school graduates
increased from 1992 to 2008? Explain. Did real wages of college
graduates increase? Did the gap between earnings of college and
highschool graduates increase? Explain, using appropriate estimates,
confidence intervals, and test statistics.

\textbf{Header for the R script}

Start a new R script, copy/paste the header below and save it to
\texttt{Dropbox\textbackslash{}EC282\textbackslash{}Assignment1} or a
similar path that you created for this homework assignment. Run the R
script and make sure that you have the data \texttt{df1} in your
enviroment. Conduct the analysis below the header.

\begin{Shaded}
\begin{Highlighting}[]
\NormalTok{###############################################################################}
\CommentTok{# list the packages we need and loads them, installs them automatically if we don't have them}
\CommentTok{# add any package that you need to the list  }
\NormalTok{need <-}\StringTok{ }\KeywordTok{c}\NormalTok{(}\StringTok{'glue'}\NormalTok{, }\StringTok{'dplyr'}\NormalTok{,}\StringTok{'readxl'}\NormalTok{, }\StringTok{'MASS'}\NormalTok{, }\StringTok{'ggplot2'}\NormalTok{,}\StringTok{'tidyr'}\NormalTok{,}\StringTok{'AER'}\NormalTok{,}\StringTok{'scales'}\NormalTok{,}\StringTok{'mvtnorm'}\NormalTok{, }
          \StringTok{'stargazer'}\NormalTok{,}\StringTok{'httr'}\NormalTok{)}

\NormalTok{have <-}\StringTok{ }\NormalTok{need }\OperatorTok\StringTok{ }\KeywordTok{rownames}\NormalTok{(}\KeywordTok{installed.packages}\NormalTok{()) }
\ControlFlowTok{if}\NormalTok{(}\KeywordTok{any}\NormalTok{(}\OperatorTok{!}\NormalTok{have)) }\KeywordTok{install.packages}\NormalTok{(need[}\OperatorTok{!}\NormalTok{have]) }
\KeywordTok{invisible}\NormalTok{(}\KeywordTok{lapply}\NormalTok{(need, library, }\DataTypeTok{character.only=}\NormalTok{T)) }

\CommentTok{# Save the R script to the assignment 1 folder before this}
\CommentTok{# To set up the working directory}
\KeywordTok{getwd}\NormalTok{()}
\KeywordTok{setwd}\NormalTok{(}\KeywordTok{getwd}\NormalTok{()) }\CommentTok{#change getwd() here is you need to set a different working directory}


\CommentTok{#this clears the workspace}
\KeywordTok{rm}\NormalTok{(}\DataTypeTok{list =} \KeywordTok{ls}\NormalTok{()) }
\CommentTok{#this sets the random number generator seed to my birthday for replication}
\KeywordTok{set.seed}\NormalTok{(}\DecValTok{06061983}\NormalTok{)}
\NormalTok{###############################################################################}
\CommentTok{#get the data url }
\NormalTok{df1.url <-}\StringTok{ 'https://wps.pearsoned.com/wps/media/objects/11422/11696965/empirical/empex_tb/cps92_08.xlsx'}
\CommentTok{#download the data }
\KeywordTok{GET}\NormalTok{(df1.url, }\KeywordTok{write_disk}\NormalTok{(tdf <-}\StringTok{ }\KeywordTok{tempfile}\NormalTok{(}\DataTypeTok{fileext =} \StringTok{".xlsx"}\NormalTok{)))}
\CommentTok{#check if it worked}
\NormalTok{df1 <-}\StringTok{ }\KeywordTok{read_excel}\NormalTok{(tdf)}
\KeywordTok{head}\NormalTok{(df1)}

\CommentTok{#CONDUCT THE ANALYSIS BELOW}
\end{Highlighting}
\end{Shaded}

\chapter{Homework Assignment II}\label{homework-assignment-ii}

\textbf{Deadline}: March 8, 2020

\textbf{Source:} Stock and Watson, 3rd Updated Edition. Exercises 4.1
and 4.2

\textbf{Data description:} You can find the data descriptions for
Question I
\href{https://wps.pearsoned.com/wps/media/objects/11422/11696965/empirical/empex_tb/CPS08_Description.pdf}{here}
and for Question II
\href{https://wps.pearsoned.com/wps/media/objects/11422/11696965/empirical/empex_tb/TeachingRatings_Description.pdf}{here}.

\textbf{Question I}

\textbf{a.} Run a regression of average hourly earnings (\texttt{ahe})
on \texttt{age}. Report the estimated intercept and the slope. Interpret
each of the estimated coefficients.

\textbf{b.} Bob is a 26-year-old worker. Predict Bob's earnings using
the estimated regression. Alexis is a 30-year-old worker. Predict Alex's
earnings using the estimated regression.

\textbf{c.} Does age explain a large fraction of the variation in
earnings across individuals? Explain.

\textbf{Question II}

\textbf{a.} Construct a scatterplot of average course evaluations --
\texttt{course\_eval} on the professor's \texttt{beauty}. Interpret the
relationship based on your graph.

\textbf{b.} Run a regression of average course evaluations
\texttt{course\_eval} on the professor's beauty \texttt{beauty}. Report
the estimated intercept and the slope. Report the sample means of
\texttt{beauty} and \texttt{course\_eval}. Explain why the estimated
intercept is equal to the sample mean of \texttt{course\_eval}.

\textbf{c.} Predict a course evaluation for a professor whose
\texttt{beauty} is one standard deviation above the average. Compare
this to the estimated intercept in (b) and explain the difference.

\textbf{d.} Interpret the size of the slope coefficient in (b). Is the
estimated ``effect'' of beauty on course evaluations large or small?

\textbf{Header for the R script}

Start a new R script, copy/paste the header below and save it to
\texttt{Dropbox\textbackslash{}EC282\textbackslash{}Assignment2} or a
similar path that you created for this homework assignment. Run the R
script and make sure that you have the data sets \texttt{df1} and
\texttt{df2} in your enviroment. Conduct the analysis below the header.

\begin{Shaded}
\begin{Highlighting}[]
\NormalTok{###############################################################################}
\CommentTok{# list the packages we need and loads them, installs them automatically if we don't have them}
\CommentTok{# add any package that you need to the list  }
\NormalTok{need <-}\StringTok{ }\KeywordTok{c}\NormalTok{(}\StringTok{'glue'}\NormalTok{, }\StringTok{'dplyr'}\NormalTok{,}\StringTok{'readxl'}\NormalTok{, }\StringTok{'MASS'}\NormalTok{, }\StringTok{'ggplot2'}\NormalTok{,}\StringTok{'tidyr'}\NormalTok{,}\StringTok{'AER'}\NormalTok{,}\StringTok{'scales'}\NormalTok{,}\StringTok{'mvtnorm'}\NormalTok{, }
          \StringTok{'stargazer'}\NormalTok{,}\StringTok{'httr'}\NormalTok{)}

\NormalTok{have <-}\StringTok{ }\NormalTok{need }\OperatorTok\StringTok{ }\KeywordTok{rownames}\NormalTok{(}\KeywordTok{installed.packages}\NormalTok{()) }
\ControlFlowTok{if}\NormalTok{(}\KeywordTok{any}\NormalTok{(}\OperatorTok{!}\NormalTok{have)) }\KeywordTok{install.packages}\NormalTok{(need[}\OperatorTok{!}\NormalTok{have]) }
\KeywordTok{invisible}\NormalTok{(}\KeywordTok{lapply}\NormalTok{(need, library, }\DataTypeTok{character.only=}\NormalTok{T)) }

\CommentTok{# Save the R script to the assignment 1 folder before this}
\CommentTok{# To set up the working directory}
\KeywordTok{getwd}\NormalTok{()}
\KeywordTok{setwd}\NormalTok{(}\KeywordTok{getwd}\NormalTok{()) }\CommentTok{#change getwd() here is you need to set a different working directory}


\CommentTok{#this clears the workspace}
\KeywordTok{rm}\NormalTok{(}\DataTypeTok{list =} \KeywordTok{ls}\NormalTok{()) }
\CommentTok{#this sets the random number generator seed to my birthday for replication}
\KeywordTok{set.seed}\NormalTok{(}\DecValTok{06061983}\NormalTok{)}
\NormalTok{###############################################################################}
\CommentTok{#get the data urls }
\NormalTok{df1.url <-}\StringTok{ 'https://wps.pearsoned.com/wps/media/objects/11422/11696965/empirical/empex_tb/cps08.xlsx'}
\NormalTok{df2.url <-}\StringTok{ 'https://wps.pearsoned.com/wps/media/objects/11422/11696965/empirical/empex_tb/TeachingRatings.xls'}
\CommentTok{#download the data }
\KeywordTok{GET}\NormalTok{(df1.url, }\KeywordTok{write_disk}\NormalTok{(tdf1 <-}\StringTok{ }\KeywordTok{tempfile}\NormalTok{(}\DataTypeTok{fileext =} \StringTok{".xlsx"}\NormalTok{)))}
\KeywordTok{GET}\NormalTok{(df2.url, }\KeywordTok{write_disk}\NormalTok{(tdf2 <-}\StringTok{ }\KeywordTok{tempfile}\NormalTok{(}\DataTypeTok{fileext =} \StringTok{".xls"}\NormalTok{)))}
\CommentTok{#check if it worked}
\NormalTok{df1 <-}\StringTok{ }\KeywordTok{read_excel}\NormalTok{(tdf1)}
\NormalTok{df2 <-}\StringTok{ }\KeywordTok{read_excel}\NormalTok{(tdf2)}
\KeywordTok{head}\NormalTok{(df1)}
\KeywordTok{head}\NormalTok{(df2)}

\CommentTok{#CONDUCT THE ANALYSIS BELOW}
\end{Highlighting}
\end{Shaded}

\chapter{Homework Assignment III}\label{homework-assignment-iii}

\textbf{Deadline}: March 22, 2020

\textbf{Source:} Stock and Watson, 3rd Updated Edition. Exercises 5.1
and 5.2

\textbf{Data description:} You can find the data descriptions for
Question I
\href{https://wps.pearsoned.com/wps/media/objects/11422/11696965/empirical/empex_tb/CPS92_08_Description.pdf}{here}
and for Question II
\href{https://wps.pearsoned.com/wps/media/objects/11422/11696965/empirical/empex_tb/TeachingRatings_Description.pdf}{here}.

\textbf{Question I}

\textbf{a.} Run a regression of average hourly earnings \texttt{ahe} on
\texttt{age} and report the regression output.

\textbf{b.} Is the estimated coefficient significant? That is, can you
reject the null hypothesis \(H_0: \beta_1 = 0\) versus a two-sided
alternative at the 10\%, 5\%, or 1\% significance level? What is the
\(p\)-value associated with coefficient's \(t\)-statistic?

\textbf{c.} Construct a 95\% confidence interval for the slope
coefficient.

\textbf{d.} Repeat (a) and (b) only using the data for college
graduates.

\textbf{Question II}

\textbf{a.} Run a regression of \texttt{course\_eval} on \texttt{beauty}
and report the regression output.

\textbf{b.} Is the estimated coefficient significant? That is, can you
reject the null hypothesis \(H_0: \beta_1 = 0\) versus a two-sided
alternative at the 10\%, 5\%, or 1\% significance level? What is the
\(p\)-value associated with coefficient's \(t\)-statistic?

\textbf{Header for the R script}

Start a new R script, copy/paste the header below and save it to
\texttt{Dropbox\textbackslash{}EC282\textbackslash{}Assignment3} or a
similar path that you created for this homework assignment. Run the R
script and make sure that you have the data sets \texttt{df1} and
\texttt{df2} in your enviroment. Conduct the analysis below the header.

\begin{Shaded}
\begin{Highlighting}[]
\NormalTok{###############################################################################}
\CommentTok{# list the packages we need and loads them, installs them automatically if we don't have them}
\CommentTok{# add any package that you need to the list  }
\NormalTok{need <-}\StringTok{ }\KeywordTok{c}\NormalTok{(}\StringTok{'glue'}\NormalTok{, }\StringTok{'dplyr'}\NormalTok{,}\StringTok{'readxl'}\NormalTok{, }\StringTok{'MASS'}\NormalTok{, }\StringTok{'ggplot2'}\NormalTok{,}\StringTok{'tidyr'}\NormalTok{,}\StringTok{'AER'}\NormalTok{,}\StringTok{'scales'}\NormalTok{,}\StringTok{'mvtnorm'}\NormalTok{, }
          \StringTok{'stargazer'}\NormalTok{,}\StringTok{'httr'}\NormalTok{)}

\NormalTok{have <-}\StringTok{ }\NormalTok{need }\OperatorTok\StringTok{ }\KeywordTok{rownames}\NormalTok{(}\KeywordTok{installed.packages}\NormalTok{()) }
\ControlFlowTok{if}\NormalTok{(}\KeywordTok{any}\NormalTok{(}\OperatorTok{!}\NormalTok{have)) }\KeywordTok{install.packages}\NormalTok{(need[}\OperatorTok{!}\NormalTok{have]) }
\KeywordTok{invisible}\NormalTok{(}\KeywordTok{lapply}\NormalTok{(need, library, }\DataTypeTok{character.only=}\NormalTok{T)) }

\CommentTok{# Save the R script to the assignment 1 folder before this}
\CommentTok{# To set up the working directory}
\KeywordTok{getwd}\NormalTok{()}
\KeywordTok{setwd}\NormalTok{(}\KeywordTok{getwd}\NormalTok{()) }\CommentTok{#change getwd() here is you need to set a different working directory}


\CommentTok{#this clears the workspace}
\KeywordTok{rm}\NormalTok{(}\DataTypeTok{list =} \KeywordTok{ls}\NormalTok{()) }
\CommentTok{#this sets the random number generator seed to my birthday for replication}
\KeywordTok{set.seed}\NormalTok{(}\DecValTok{06061983}\NormalTok{)}
\NormalTok{###############################################################################}
\CommentTok{#get the data urls }
\NormalTok{df1.url <-}\StringTok{ 'https://wps.pearsoned.com/wps/media/objects/11422/11696965/empirical/empex_tb/cps08.xlsx'}
\NormalTok{df2.url <-}\StringTok{ 'https://wps.pearsoned.com/wps/media/objects/11422/11696965/empirical/empex_tb/TeachingRatings.xls'}
\CommentTok{#download the data }
\KeywordTok{GET}\NormalTok{(df1.url, }\KeywordTok{write_disk}\NormalTok{(tdf1 <-}\StringTok{ }\KeywordTok{tempfile}\NormalTok{(}\DataTypeTok{fileext =} \StringTok{".xlsx"}\NormalTok{)))}
\KeywordTok{GET}\NormalTok{(df2.url, }\KeywordTok{write_disk}\NormalTok{(tdf2 <-}\StringTok{ }\KeywordTok{tempfile}\NormalTok{(}\DataTypeTok{fileext =} \StringTok{".xls"}\NormalTok{)))}
\CommentTok{#check if it worked}
\NormalTok{df1 <-}\StringTok{ }\KeywordTok{read_excel}\NormalTok{(tdf1)}
\NormalTok{df2 <-}\StringTok{ }\KeywordTok{read_excel}\NormalTok{(tdf2)}
\KeywordTok{head}\NormalTok{(df1)}
\KeywordTok{head}\NormalTok{(df2)}

\CommentTok{#CONDUCT THE ANALYSIS BELOW}
\end{Highlighting}
\end{Shaded}

\chapter{Homework Assignment IV}\label{homework-assignment-iv}

\textbf{Deadline}: April 12, 2020

\textbf{Source:} Stock and Watson, 3rd Updated Edition. Exercises 6.1
and 6.3.

\textbf{Data description:} You can find the data descriptions for
Question I
\href{https://wps.pearsoned.com/wps/media/objects/11422/11696965/empirical/empex_tb/TeachingRatings_Description.pdf}{here}
and for Question II
\href{https://wps.pearsoned.com/wps/media/objects/11422/11696965/empirical/empex_tb/Growth_Description.pdf}{here}.

\textbf{Question I}

\textbf{a.} Run a regression of \texttt{course\_eval} on
\texttt{beauty}. On a second regression, add the following control
variables: \texttt{intro}, \texttt{onecredit}, \texttt{female},
\texttt{minority} and \texttt{nnenglish}. Report both regression
outputs. Compare the estimated ``effect'' of \texttt{beauty} in the
first to the second regression? Does the first estimated slope change
substantially after adding the control variables to the model? What does
that indicate?

\textbf{b.} Predict the outcome for a black male professor with average
beauty and is a native English speaker. He teaches a three-credit
upper-divisiocourse.

\textbf{Question II}

\textbf{a.} Drop the observation for Malta from the analysis data set.

\textbf{b.} Construct a table that shows the sample mean, standard
deviation, and minimum and maximum values for the series
\texttt{growth}, \texttt{tradeshare}, \texttt{yearsschool},
\texttt{oil}, \texttt{rev\_coups}, \texttt{assasinations}, and
\texttt{rgdp60}.

\textbf{c.} Run a regression of \texttt{growth} on \texttt{tradeshare},
\texttt{yearsschool}, \texttt{rev\_coups}, \texttt{assasinations}, and
\texttt{rgdp60}. Report the regression results in a table and interpret
the coefficient on \texttt{rev\_coups}.

\textbf{d.} Use the regression results to predict the average annual
growth rate for a country that has average values for all regressors.

\textbf{e.} Include \texttt{oil} to regression (c) and interpret any
\textbf{major} changes in regression (c).

\textbf{Header for the R script}

Start a new R script, copy/paste the header below and save it to
\texttt{Dropbox\textbackslash{}EC282\textbackslash{}Assignment4} or a
similar path that you created for this homework assignment. Run the R
script and make sure that you have the data sets \texttt{df1} and
\texttt{df2} in your enviroment. Conduct the analysis below the header.

\begin{Shaded}
\begin{Highlighting}[]
\NormalTok{###############################################################################}
\CommentTok{# list the packages we need and loads them, installs them automatically if we don't have them}
\CommentTok{# add any package that you need to the list  }
\NormalTok{need <-}\StringTok{ }\KeywordTok{c}\NormalTok{(}\StringTok{'glue'}\NormalTok{, }\StringTok{'dplyr'}\NormalTok{,}\StringTok{'readxl'}\NormalTok{, }\StringTok{'MASS'}\NormalTok{, }\StringTok{'ggplot2'}\NormalTok{,}\StringTok{'tidyr'}\NormalTok{,}\StringTok{'AER'}\NormalTok{,}\StringTok{'scales'}\NormalTok{,}\StringTok{'mvtnorm'}\NormalTok{, }
          \StringTok{'stargazer'}\NormalTok{,}\StringTok{'httr'}\NormalTok{)}

\NormalTok{have <-}\StringTok{ }\NormalTok{need }\OperatorTok\StringTok{ }\KeywordTok{rownames}\NormalTok{(}\KeywordTok{installed.packages}\NormalTok{()) }
\ControlFlowTok{if}\NormalTok{(}\KeywordTok{any}\NormalTok{(}\OperatorTok{!}\NormalTok{have)) }\KeywordTok{install.packages}\NormalTok{(need[}\OperatorTok{!}\NormalTok{have]) }
\KeywordTok{invisible}\NormalTok{(}\KeywordTok{lapply}\NormalTok{(need, library, }\DataTypeTok{character.only=}\NormalTok{T)) }

\CommentTok{# Save the R script to the assignment 1 folder before this}
\CommentTok{# To set up the working directory}
\KeywordTok{getwd}\NormalTok{()}
\KeywordTok{setwd}\NormalTok{(}\KeywordTok{getwd}\NormalTok{()) }\CommentTok{#change getwd() here is you need to set a different working directory}


\CommentTok{#this clears the workspace}
\KeywordTok{rm}\NormalTok{(}\DataTypeTok{list =} \KeywordTok{ls}\NormalTok{()) }
\CommentTok{#this sets the random number generator seed to my birthday for replication}
\KeywordTok{set.seed}\NormalTok{(}\DecValTok{06061983}\NormalTok{)}
\NormalTok{###############################################################################}
\CommentTok{#get the data urls }
\NormalTok{df1.url <-}\StringTok{ 'https://wps.pearsoned.com/wps/media/objects/11422/11696965/empirical/empex_tb/TeachingRatings.xls'}
\NormalTok{df2.url <-}\StringTok{ 'https://wps.pearsoned.com/wps/media/objects/11422/11696965/empirical/empex_tb/Growth.xls'}

\CommentTok{#download the data }
\KeywordTok{GET}\NormalTok{(df1.url, }\KeywordTok{write_disk}\NormalTok{(tdf1 <-}\StringTok{ }\KeywordTok{tempfile}\NormalTok{(}\DataTypeTok{fileext =} \StringTok{".xls"}\NormalTok{)))}
\KeywordTok{GET}\NormalTok{(df2.url, }\KeywordTok{write_disk}\NormalTok{(tdf2 <-}\StringTok{ }\KeywordTok{tempfile}\NormalTok{(}\DataTypeTok{fileext =} \StringTok{".xls"}\NormalTok{)))}

\CommentTok{#check if it worked}
\NormalTok{df1 <-}\StringTok{ }\KeywordTok{read_excel}\NormalTok{(tdf1)}
\NormalTok{df2 <-}\StringTok{ }\KeywordTok{read_excel}\NormalTok{(tdf2)}
\KeywordTok{head}\NormalTok{(df1)}
\KeywordTok{head}\NormalTok{(df2)}

\CommentTok{#CONDUCT THE ANALYSIS BELOW}
\end{Highlighting}
\end{Shaded}

\chapter{Homework Assignment V}\label{homework-assignment-v}

\textbf{Deadline}: April 26, 2020

\textbf{Source:} Stock and Watson, 3rd Updated Edition. Exercise 8.1

\textbf{Data description:} You can find the data descriptions for
Question I
\href{https://wps.pearsoned.com/wps/media/objects/11422/11696965/empirical/empex_tb/CPS08_Description.pdf}{here}.

\textbf{QUESTION I}

\textbf{a.} Run a regression of average hourly earnings (\texttt{ahe})
on \texttt{age},\texttt{female}, and \texttt{bachelor} and report the
output. If \texttt{age} increases from 25 to 26, how are earnings
expected to change? If \texttt{age} increases from 33 to 34, how are
earnings expected to change?

\textbf{b.} Run a regression of the \textbf{logarithm} of average hourly
earnings (\texttt{ln\_ahe}) on \texttt{age},\texttt{female}, and
\texttt{bachelor} and report the output. If \texttt{age} increases from
25 to 26, how are earnings expected to change? If \texttt{age} increases
from 33 to 34, how are earnings expected to change?

\textbf{c.} Run a regression of the \textbf{logarithm} of average hourly
earnings (\texttt{ln\_ahe}) on the \textbf{logarithm} of
\texttt{ln\_age},\texttt{female}, and \texttt{bachelor} and report the
output. If \texttt{age} increases from 25 to 26, how are earnings
expected to change? If \texttt{age} increases from 33 to 34, how are
earnings expected to change?

\textbf{d.} Run a regression of the \textbf{logarithm} of average hourly
earnings (\texttt{ln\_ahe}) on \texttt{age}, square of age
(\texttt{age\_sq}), \texttt{female}, and \texttt{bachelor} and report
the output. If \texttt{age} increases from 25 to 26, how are earnings
expected to change? If \texttt{age} increases from 33 to 34, how are
earnings expected to change?

\textbf{e.} Comparing the regression results from (a),(b),(c),(d),
choose one of the empirical models based on economic theory. Briefly
explain why you choose the preferred model.

\textbf{f.} Run a regression of the \textbf{logarithm} of average hourly
earnings (\texttt{ln\_ahe}) on \texttt{age}, square of age
(\texttt{age\_sq}), \texttt{female}, and \texttt{bachelor} and the
interaction term \texttt{female} \(\times\) \texttt{bachelor}. Report
the output. Consider the following individuals:

\begin{itemize}
\tightlist
\item
  Alexis: 30-year-old female with a bachelor's degree.
\item
  Jane: 30-year-old female with a high school degree.
\item
  Bob: 30-year-old male with a bachelor degree.
\item
  Jim: 30-year-old male with a high school degree.
\end{itemize}

Using the regression results, predict \texttt{ln\_ahe} and \texttt{ahe}
for each individual. Calculate the college premium (predicted difference
in log wage) for females and college premium for males. Is the college
premium differ for female vs.~males? Explain.

\textbf{g.} Is the effect of \texttt{age} on earnings different for man
than women? Using the variables \texttt{ln\_ahe}, \texttt{female},
\texttt{bachelor}, and \texttt{age}. Specify and estimate a regression
that you can use to answer this question. Report the regression and
explain your results.

\textbf{Header for the R script}

Start a new R script, copy/paste the header below and save it to
\texttt{Dropbox\textbackslash{}EC282\textbackslash{}Assignment5} or a
similar path that you created for this homework assignment. Run the R
script and make sure that you have the data sets \texttt{df1} and
\texttt{df2} in your enviroment. Conduct the analysis below the header.

\begin{Shaded}
\begin{Highlighting}[]
\NormalTok{###############################################################################}
\CommentTok{# list the packages we need and loads them, installs them automatically if we don't have them}
\CommentTok{# add any package that you need to the list  }
\NormalTok{need <-}\StringTok{ }\KeywordTok{c}\NormalTok{(}\StringTok{'glue'}\NormalTok{, }\StringTok{'dplyr'}\NormalTok{,}\StringTok{'readxl'}\NormalTok{, }\StringTok{'MASS'}\NormalTok{, }\StringTok{'ggplot2'}\NormalTok{,}\StringTok{'tidyr'}\NormalTok{,}\StringTok{'AER'}\NormalTok{,}\StringTok{'scales'}\NormalTok{,}\StringTok{'mvtnorm'}\NormalTok{, }
          \StringTok{'stargazer'}\NormalTok{,}\StringTok{'httr'}\NormalTok{)}

\NormalTok{have <-}\StringTok{ }\NormalTok{need }\OperatorTok\StringTok{ }\KeywordTok{rownames}\NormalTok{(}\KeywordTok{installed.packages}\NormalTok{()) }
\ControlFlowTok{if}\NormalTok{(}\KeywordTok{any}\NormalTok{(}\OperatorTok{!}\NormalTok{have)) }\KeywordTok{install.packages}\NormalTok{(need[}\OperatorTok{!}\NormalTok{have]) }
\KeywordTok{invisible}\NormalTok{(}\KeywordTok{lapply}\NormalTok{(need, library, }\DataTypeTok{character.only=}\NormalTok{T)) }

\CommentTok{# Save the R script to the assignment 1 folder before this}
\CommentTok{# To set up the working directory}
\KeywordTok{getwd}\NormalTok{()}
\KeywordTok{setwd}\NormalTok{(}\KeywordTok{getwd}\NormalTok{()) }\CommentTok{#change getwd() here is you need to set a different working directory}


\CommentTok{#this clears the workspace}
\KeywordTok{rm}\NormalTok{(}\DataTypeTok{list =} \KeywordTok{ls}\NormalTok{()) }
\CommentTok{#this sets the random number generator seed to my birthday for replication}
\KeywordTok{set.seed}\NormalTok{(}\DecValTok{06061983}\NormalTok{)}
\NormalTok{###############################################################################}
\CommentTok{#get the data urls }
\NormalTok{df1.url <-}\StringTok{ 'https://wps.pearsoned.com/wps/media/objects/11422/11696965/empirical/empex_tb/cps08.xlsx'}
\CommentTok{#download the data }
\KeywordTok{GET}\NormalTok{(df1.url, }\KeywordTok{write_disk}\NormalTok{(tdf1 <-}\StringTok{ }\KeywordTok{tempfile}\NormalTok{(}\DataTypeTok{fileext =} \StringTok{".xlsx"}\NormalTok{)))}
\CommentTok{#check if it worked}
\NormalTok{df1 <-}\StringTok{ }\KeywordTok{read_excel}\NormalTok{(tdf1)}
\KeywordTok{head}\NormalTok{(df1)}


\CommentTok{#CONDUCT THE ANALYSIS BELOW}
\end{Highlighting}
\end{Shaded}

\chapter{Lecture notes, updates, and
corrections:}\label{lecture-notes-updates-and-corrections}

I will periodically update this page with classroom lecture notes,
answer to your questions, additional code, tips, etc.

\bibliography{book.bib,packages.bib}

\end{document}
